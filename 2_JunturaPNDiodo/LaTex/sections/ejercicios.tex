\documentclass[../main.tex]{subfiles}

\begin{document}
	
	\subsection{Ejercicio 2.1}

	Para una juntura abrupta idealizada de Si a T=300K, en equilibrio térmico, con $N_A = 1 \cdot 10^{14} cm^{-3}$  y  $N_D = 5 \cdot 10^{13} cm^{-3}$ , calcular:
	
		a) El potencial de contacto.
		
		b) El ancho  $l$  de la juntura en la zona de carga espacial.
		
		c) Las longitudes $l_n$ y $l_p$.
		
		d) El campo eléctrico máximo $E_{max}$.
		
		e) La capacidad específica de juntura $C^{'}_{J_0}$.
	
	\bigskip
	
	\paragraph{a)}
		
	La ecuación que permite calcular el potencial de contacto es la siguiente:
	
	\begin{equation}
		\psi_0 = \frac{k \cdot T}{q_e} \cdot ln \left( \frac{N_A \cdot N_D}{n_i^2(T)}\right) 
	\end{equation} 
	
	Reemplazando los valores del problema:
	
	\begin{equation}
		\psi_0 = \frac{1.38 \cdot 10^{-23} \frac{J}{K} \cdot 300 K}{1.6 \cdot 10^{-19} C} \cdot ln \left( \frac{1 \cdot 10^{14} cm^{-3} \cdot 5 \cdot 10^{13} cm^{-3}}{{\left( 1.5 \cdot 10^{10} cm^{-3}\right) }^2}\right) 
	\end{equation} 
	
	\begin{equation}
		\fbox{
			$\psi_0 \approx 0.459 V \approx 459 mV$
			}
	\end{equation} 
	
	\paragraph{b)}
	
	La ecuación que permite calcular el ancho de juntura es la siguiente:
	
	
	
	
\end{document}